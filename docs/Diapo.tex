\documentclass{beamer}

\usepackage[utf8]{inputenc}
\usepackage{ascii}
\usepackage[T1]{fontenc}
\usepackage[french]{babel}
\usepackage{mathtools}
\usepackage{stmaryrd}
\usepackage{color}
\usepackage{listings}
\usepackage{tikz}
\usepackage{graphicx}

\usetheme{Pittsburgh}
\beamertemplatenavigationsymbolsempty
\beamertemplatelargeframetitle
\beamertemplateheadempty
\beamertemplatefootpagenumber

\renewcommand{\ttfamily}{\asciifamily}

\lstset{
  escapeinside={\%*}{*},
  language=Python,
  morekeywords={as},
  basicstyle=\ttfamily\scriptsize,
  tabsize=2,
  breakatwhitespace=true,
  breaklines=true,
  showstringspaces=false,
  showtabs=false,
  numbers=left,
  numbersep=5pt,
  stepnumber=1,
  keywordstyle=\color{purple},
  identifierstyle=\color{blue},
  stringstyle=\color[HTML]{008800},
  commentstyle=\color[gray]{0.25},
  numberstyle=\tiny\color{gray}
}

\title[TIPE]{Modélisation de l'évolution de feux de forêts\\à l'aide d'automates cellulaires}
\author{32366 BERTHOUD Tanguy}
\date{2019}

\begin{document}
  \frame[plain]{
    \titlepage
  }
  \AtBeginSection[]{\frame[plain]{\tableofcontents[current]}}
  % 1 - OBJETS DU PROGRAMME
  \section[Objets]{Structure du programme}
  \frame{
    \frametitle{Structure du programme}
    \tikz{
      \filldraw[fill=green] (0,0) rectangle +(1,-1);
      \draw[step=1] (0,0) grid +(1,-1);
      \node[anchor=west] at (2,-0.5) {\textbf{Tree}\((q,T,d,h)\)};

      \draw[thin] (0,-2) -- (0,-1);
      \draw[thin] (0.25,-2) -- (1,-1);

      \filldraw[fill=green] (0,-2) rectangle +(1,-1);
      \draw[step=0.25] (0,-2) grid +(1,-1);
      \node[anchor=west] at (2,-2.5) {\textbf{Forest}\((\mathcal{F},n,\Delta,W)\)};

      \draw[thin] (0,-4) -- (0,-3);
      \draw[thin] (1,-4) -- (1,-3);

      \draw[very thick] (0.25,-5.25) -- ++(-0.5,0) -- ++(0,1.5) -- ++(0.5,0);
      \foreach \i/\itext [evaluate=\i as \ii using \i+0.5,evaluate=\i as \iii using \i+1.12] in {0/0,1.25/1,2.50/2} {
        \filldraw[fill=green] (\i,-4) rectangle +(1,-1);
        \draw[step=0.25] (\i,-4) grid +(1,-1);
        \node at (\ii,-4.5) {\textbf{\itext}};
        \node at (\iii,-4.5) {,};
      };
      \node at (4,-4.5) {\dots};
      \draw[very thick] (4,-5.25) -- ++(0.5,0) -- ++(0,1.5) -- ++(-0.5,0);
      \node[anchor=west] at (5,-4.5) {\textbf{Wildfire}\((\mathcal{H},\Lambda,\tau)\)};
    }
  }
  % 1.1 - Tree
  \subsection[Tree]{Objet Tree}
  \frame[containsverbatim]{
    \frametitle{Structure du programme}
    \framesubtitle{Objet Tree}
    Un objet \verb!Tree! représente un arbre \((q,T,d,h) \in \mathbb{T}\) :
    \begin{description}
      \item[state] \(q \in \{0,1,2\}\)
      \item[temperature] \(T \in \mathbb{R}_{+}\)
      \item[grass] \(d \in \mathbb{R}_{+}\)
      \item[humidity] \(h \in [0;1]\)
    \end{description}
  }
  % 1.2 - Forest
  \subsection[Forest]{Objet Forest}
  \frame[containsverbatim]{
  \frametitle{Structure du programme}
  \framesubtitle{Objet Forest}
    Un objet \verb!Forest! représente une forêt \((\mathcal{F},n,\Delta,W) \in \mathbb{F}\) :
    \begin{description}
      \item[Forest] \(\mathcal{F} \in M_{n}(\mathbb{T})\) contient les objets \verb!Tree!
      \item[dim] \(n \in \mathbb{N}^{*}\)
      \item[start] \(\Delta \subset \llbracket 1,n \rrbracket ^{2}\)
      \item[wind] \(W \in \llbracket -1,7 \rrbracket\)
    \end{description}
  }
  % 1.3 - Wildfire
  \subsection[Wildfire]{Objet Wildfire}
  \frame[containsverbatim]{
  \frametitle{Structure du programme}
  \framesubtitle{Objet Wildfire}
    Un objet \verb!Wildfire! est un historique des forêts calculées \((\mathcal{H},\Lambda,\tau) \in \mathbb{W}\) :
    \begin{description}
      \item[History] \(\mathcal{H} \in \mathbb{F}^{\mathbb{N}}\) contient les objets \verb!Forest!
      \item[model] \(\Lambda : \mathbb{W} \longrightarrow \mathbb{F}\)
      \item[dt] \(\tau \in \mathbb{R}_{+}^{*}\)
    \end{description}
  }
  % 2 - FONCTIONS AUXILIAIRES
  \section[Auxiliaires]{Fonctions auxiliaires}
  % 2.1 - neighbours
  \subsection{Tree.neighbours}
  \frame[containsverbatim]{
    \frametitle{Fonctions auxiliaires}
    \framesubtitle{Tree.neighbours}
    \begin{columns}
      \column{.4\textwidth}
      \tikz[scale=0.75,every node/.style={scale=0.75}]{
        \filldraw[orange] (1,-1) rectangle +(3,-3);
        \filldraw[green] (2,-2) rectangle +(1,-1);
        \node at (2.5,-2.5) {\(A\)};
        \draw[step=1] (-0.1,0.1) grid +(5.2,-5.2);
        \foreach \pos/\text in {(2.5,-1.5)/N,(3.5,-1.5)/NE,(3.5,-2.5)/E,(3.5,-3.5)/SE,(2.5,-3.5)/S,(1.5,-3.5)/SW,(1.5,-2.5)/W,(1.5,-1.5)/NW}
          \node at \pos {\textbf{\text}};
        \draw[->,thin] (2.5,-1.75) arc [start angle=90,end angle=-225,radius=0.75];
      }

      \column{.6\textwidth}
      \textbf{Voisinage de \textit{Moore} trié}
      \bigbreak
      Soit \(F(\mathcal{F},n,\Delta,W) \in \mathbb{F}\).\\
      Soit \(A(q,T,d,h) = [\mathcal{F}]_{i,j} \in \mathbb{T}\).
      \[
        \boxed{V(A) \in (\mathbb{T} \cup \{\text{\verb!None!}\})^{8}}
      \]
    \end{columns}
  }
  % 2.2 - windDistCalc
  \subsection{Tree.windDirCalc}
  \frame{
    \frametitle{Fonctions auxiliaires}
    \framesubtitle{Tree.windDirCalc}
    \begin{columns}
      \column{.4\textwidth}
      \tikz[scale=0.75,every node/.style={scale=0.75}]{
        \filldraw[red] (1,-3) rectangle +(1,-1);
        \node[anchor=north] at (1.5,-2.9) {\(S\)};
        \filldraw[green] (4,-1) rectangle +(1,-1);
        \node[anchor=north west] at (4,-1) {\(A\)};
        \draw[step=1] (-0.1,0.1) grid +(5.2,-5.2);
        \draw[blue,dashed,thick] (3.1,-5.1) -- (-0.1,-1.9) node[anchor=south west] {\(\Delta_{W}\)};
        \draw[->,blue,very thick] (1,-3) -- +(1,-1) node[anchor=north east] {\(\overrightarrow{W}\)};
        \draw[orange,dashed,thick] (-0.1,-4.58) -- (5.1,-1.1);
        \draw[->,orange,very thick] (1.5,-3.5) -- +(3,2) node[anchor=north west] {\(\overrightarrow{SA}\)};
        \draw[|->,orange,very thick] (2.5,-4.5) arc [start angle=-45,end angle=30,radius=1.5];
        \node[orange,anchor=west] at (3,-3.5) {\(\varphi(S,A)\)};
      }

      \column{.6\textwidth}
      \textbf{Cosinus de l'angle \(\varphi(S,A)\)}
      \bigbreak
      Soit \(F(\mathcal{F},n,\Delta,W) \in \mathbb{F}\).\\
      {\tiny Soient \(S(q_{0},T_{0},d_{0},h_{0}) = [\mathcal{F}]_{i_{0},j_{0}}; A(q,T,d,h) = [\mathcal{F}]_{i,j} \in \mathbb{T}\).}\\
      Soit \(\overrightarrow{W} =\begin{cases}
        (-1;0) & \text{si } W = 0\\
        (-1;1) & \text{si } W = 1\\
        \vdots\\
        (-1;-1) & \text{si } W = 7
      \end{cases}\).
      \[
        \boxed{\cos{\varphi(S,A)} = \frac{\overrightarrow{W} \cdot \overrightarrow{SA}}{\|\overrightarrow{W}\| \times \|\overrightarrow{SA}\|}}
      \]
    \end{columns}
  }
  % 2.3 - physics
  \subsection{Wildfire.physics}
  \frame{
    \frametitle{Fonctions auxiliaires}
    \framesubtitle{Wildfire.physics}
    \begin{onlyenv}<1>
      \textbf{Vitesses de propagation des flammes sans et avec vent}
      \bigbreak
      Soit \(F(\mathcal{F},n,\Delta,W) \in \mathbb{F}\).\\
      Soit \(A(q,T,d,h) = [\mathcal{F}]_{i,j} \in \mathbb{T}\).
      \[
        \boxed{v_{0} : (d,h) \mapsto \frac{B\,T_{fl}^{4} \times d}{2 \sigma_{herbe} (c_{p} (T_{igni} - T_{i}) + \Delta h_{eau} \times h)}}
      \]\[
        \boxed{v_{1} : (d,h) \mapsto v_{0}(d,h) \times \varepsilon_{fl}\,H_{fl}\frac{1 + \sin{\gamma} - \cos{\gamma}}{d}}
      \]\begin{center}\begin{tabular}{| r c l || r c l |}
        \hline
        \(B\) & \(=\) & \(5,670.10^{-8}\,W.m^{-2}.K^{-4}\) & \(T_{fl}\) & \(=\) & \(1300,15\,K\)\\ \hline
        \(\sigma_{herbe}\) & \(=\) & \(0,2\,kg.m^{-2}\) & \(c_{p}\) & \(=\) & \(1200\,J.kg^{-1}.K^{-1}\)\\ \hline
        \(T_{igni}\) & \(=\) & \(573,15\,K\) & \(T_{i}\) & \(=\) & \(298,15\,K\)\\ \hline
        \(\Delta h_{eau}\) & \(=\) & \(2256\,kJ.kg^{-1}\) & \(\varepsilon_{fl}\) & \(=\) & \(0,90\)\\ \hline
        \(H_{fl}\) & \(=\) & \(2\,m\) & \(\gamma\) & \(=\) & \(20^{\circ}\)\\
        \hline
      \end{tabular}\end{center}
    \end{onlyenv}

    \begin{onlyenv}<2>
      \textbf{Temps de parcours des flammes}
      \bigbreak
      Soit \(F(\mathcal{F},n,\Delta,W) \in \mathbb{F}\).\\
      {\small Pour tout \(T(q,T,d,h) = [\mathcal{F}]_{i,j} \in \mathbb{T}\); notons \(d_{(i,j)} = d\) et \(h_{(i,j)} = h\).}\\
      {\scriptsize Pour tous \(T_{1}(q_{1},T_{1},d_{1},h_{1}) = [\mathcal{F}]_{i_{1},j_{1}}; T_{2}(q_{2},T_{2},d_{2},h_{2}) = [\mathcal{F}]_{i_{2},j_{2}} \in \mathbb{T}\); notons :} \begin{center}\begin{tabular}{c | r c l}
        \(\eta_{T_{1},T_{2}}\) : & \([0;1]\) & \(\to\) & \(\llbracket 1,n \rrbracket ^{2}\)\\
        & \(t\) & \(\mapsto\) & \((\lfloor t\,i_{1}+(1-t)i_{2} \rfloor,\lfloor t\,j_{1}+(1-t)j_{2} \rfloor)\)
      \end{tabular}\end{center}
      Soit \(A(q,T,d,h) = [\mathcal{F}]_{i,j} \in \mathbb{T}\).
      {\tiny\[
        \boxed{[\mathcal{T}]_{i,j} =\begin{cases}
          \min{\left\{ \displaystyle\int_{0}^{1} \frac{\|\overrightarrow{SA}\|\,\mathrm{d}t}{v_{0}(d_{\eta_{S,A}(t)},h_{\eta_{S,A}(t)})} \middle| S = [\mathcal{F}]_{i_{0},j_{0}} \in \Delta \right\}} & \text{si } W = -1\\
          \min{\left\{ \displaystyle\int_{0}^{1} \frac{\|\overrightarrow{SA}\|\,\mathrm{d}t}{(1,3\,+\,\cos(\overrightarrow{W},\overrightarrow{SA})) \times v_{1}(d_{\eta_{S,A}(t)},h_{\eta_{S,A}(t)})} \middle| S = [\mathcal{F}]_{i_{0},j_{0}} \in \Delta \right\}} & \text{si } W \in \llbracket 0,7 \rrbracket
        \end{cases}}
      \]}
    \end{onlyenv}

    \begin{onlyenv}<3>
      \textbf{Arbre d'origine des flammes}
      \bigbreak
      Soit \(F(\mathcal{F},n,\Delta,W) \in \mathbb{F}\).\\
      {\small Pour tout \(T(q,T,d,h) = [\mathcal{F}]_{i,j} \in \mathbb{T}\); notons \(d_{(i,j)} = d\) et \(h_{(i,j)} = h\).}\\
      {\scriptsize Pour tous \(T_{1}(q_{1},T_{1},d_{1},h_{1}) = [\mathcal{F}]_{i_{1},j_{1}}; T_{2}(q_{2},T_{2},d_{2},h_{2}) = [\mathcal{F}]_{i_{2},j_{2}} \in \mathbb{T}\); notons :} \begin{center}\begin{tabular}{c | r c l}
        \(\eta_{T_{1},T_{2}}\) : & \([0;1]\) & \(\to\) & \(\llbracket 1,n \rrbracket ^{2}\)\\
        & \(t\) & \(\mapsto\) & \((\lfloor t\,i_{1}+(1-t)i_{2} \rfloor,\lfloor t\,j_{1}+(1-t)j_{2} \rfloor)\)
      \end{tabular}\end{center}
      Soit \(A(q,T,d) = [\mathcal{F}]_{i,j} \in \mathbb{T}\).
      {\tiny\[
        \text{Soit } S = [\mathcal{F}]_{i_{0},j_{0}} \in \Delta \text{ tel que } [\mathcal{T}]_{i,j} =\begin{cases}
          \displaystyle\int_{0}^{1} \frac{\|\overrightarrow{SA}\|\,\mathrm{d}t}{v_{0}((d_{\eta_{S,A}(t)},h_{\eta_{S,A}(t)})} & \text{si } W = -1\\
          \displaystyle\int_{0}^{1} \frac{\|\overrightarrow{SA}\|\,\mathrm{d}t}{(1,3 + \cos(\overrightarrow{W},\overrightarrow{SA})) \times v_{1}(d_{\eta_{S,A}(t)},h_{\eta_{S,A}(t)})} & \text{si } W \in \llbracket 0,7 \rrbracket
        \end{cases}\text{.}
      \]}\[
        \text{Alors } \boxed{[\mathcal{O}]_{i,j} = (i_{0},j_{0})} \text{.}
      \]
    \end{onlyenv}
  }
  % 2 - MODÈLES
  \section[Modèles]{Modèles de propagation des flammes}
  % 2.1 - Modèle EMPIRIQUE
  \subsection{Premier modèle}
  \frame{
    \frametitle{Modèles de propagation des flammes}
    \framesubtitle{Premier modèle}
    Soit \(W(\mathcal{H},\Lambda,\tau) \in \mathbb{W}\).\\
    Soit \(F(\mathcal{F},n,\Delta,W) \in \mathcal{H}\). Posons \(F'(\mathcal{F}',n,\Delta,W) = \Lambda(F)\).\\
    Soit \(A(q,T,d,h) = [\mathcal{F}]_{i,j}\).
    \bigbreak
    \begin{columns}
      \column{.6\textwidth}
      \textbf{Boucle d'actualisation de \(T\)}
      \bigbreak
      \tikz[scale=0.75,every node/.style={scale=0.75}]{
        \node[anchor=south] at (0.5,0) {\(\mathcal{F}\)};
        \node[anchor=south] at (5.5,0) {\(\mathcal{F}'\)};

        \filldraw[red] (0,0) rectangle +(1,-1);
        \node at (0.5,-0.5) {\(A\)};
        \filldraw[green] (-1,-2) rectangle +(1,-1);
        \filldraw[red] (0,-2) rectangle +(1,-1);
        \node at (-0.5,-2.5) {\(A\)};
        \node at (0.5,-2.5) {\(S\)};
        \filldraw[gray] (0,-4) rectangle +(1,-1);
        \node at (0.5,-4.5) {\(A\)};
        \filldraw[red] (5,0) rectangle +(1,-1);
        \node at (5.5,-0.5) {\(A\)};
        \filldraw[green] (5,-2) rectangle +(1,-1);
        \filldraw[red] (6,-2) rectangle +(1,-1);
        \node at (5.5,-2.5) {\(A\)};
        \node at (6.5,-2.5) {\(S\)};
        \filldraw[gray] (5,-4) rectangle +(1,-1);
        \node at (5.5,-4.5) {\(A\)};

        \foreach \i in {-0.5,-2.5,-4.5} \draw[|->,thick] (1.1,\i) -- +(3.8,0);
        \node[anchor=north] at (3,-0.5) {\(T + 25 \tau\)};
        \node[anchor=north,scale=0.9] at (3,-2.5) {\(T + 15 \tau (1 + \cos(\overrightarrow{W},\overrightarrow{SA}))\)};
        \node[anchor=north,scale=0.8] at (3,-4.5) {\(T - \tau (10 + 5 \cos(\overrightarrow{W},\overrightarrow{[\mathcal{O}]_{i,j}A}))\)};

        \draw[step=1] (0,0) grid +(1,-1);
        \draw[step=1] (-1,-2) grid +(2,-1);
        \draw[step=1] (0,-4) grid +(1,-1);
        \draw[step=1] (5,0) grid +(1,-1);
        \draw[step=1] (5,-2) grid +(2,-1);
        \draw[step=1] (5,-4) grid +(1,-1);
      }

      \column{.4\textwidth}
      \textbf{Boucle d'actualisation de \(q\)}
      \bigbreak
      \tikz[scale=0.75,every node/.style={scale=0.75}]{
        \node[anchor=south] at (0.5,0) {\(\mathcal{F}\)};
        \node[anchor=south] at (4.5,0) {\(\mathcal{F}'\)};

        \filldraw[green] (0,0) rectangle +(1,-2);
        \foreach \i in {-0.5,-1.5} \node at (0.5,\i) {\(q = 0\)};
        \filldraw[red] (0,-2) rectangle +(1,-2);
        \foreach \i in {-2.5,-3.5} \node at (0.5,\i) {\(q = 1\)};
        \filldraw[gray] (0,-4) rectangle +(1,-1);
        \node at (0.5,-4.5) {\(q = 2\)};
        \filldraw[green] (4,0) rectangle +(1,-1);
        \node at (4.5,-0.5) {\(q = 0\)};
        \filldraw[red] (4,-1) rectangle +(1,-2);
        \foreach \i in {-1.5,-2.5} \node at (4.5,\i) {\(q = 1\)};
        \filldraw[gray] (4,-3) rectangle +(1,-2);
        \foreach \i in {-3.5,-4.5} \node at (4.5,\i) {\(q = 2\)};

        \foreach \i in {-0.5,...,-4.5} \draw[|->,thick] (1.1,\i) -- +(2.8,0);
        \node[anchor=north] at (2.5,-0.5) {si \(T < T_{igni}\)};
        \node[anchor=south] at (2.5,-1.5) {si \(T \geq T_{igni}\)};
        \node[anchor=north,scale=0.9] at (2.5,-2.5) {si \(T < 5000 + T_{igni}\)};
        \node[anchor=south,scale=0.9] at (2.5,-3.5) {si \(T \geq 5000 + T_{igni}\)};

        \draw[step=1] (0,0) grid +(1,-5);
        \draw[step=1] (4,0) grid +(1,-5);
      }
    \end{columns}
  }
  % 2.2 - Modèle SEMI-PHYSIQUE
  \subsection{Second modèle}
  \frame[containsverbatim]{
    \frametitle{Modèles de propagation des flammes}
    \framesubtitle{Second modèle}
    Soit \(W(\mathcal{H},\Lambda,\tau) \in \mathbb{W}\).\\
    Soit \(F(\mathcal{F},n,\Delta,W) \in \mathcal{H}\). Posons \(F'(\mathcal{F}',n,\Delta,W) = \Lambda(F)\).\\
    {\small Soit \(A(q,T,d,h) = [\mathcal{F}]_{i,j}\). Posons \(V(A)^{*} = \{T \in V(A) | T \neq \text{\verb!None!}\}\).}
    \bigbreak
    \begin{columns}
      \column{.6\textwidth}
      \textbf{Boucle d'actualisation de \(T\)}
      \bigbreak
      \tikz[scale=0.75,every node/.style={scale=0.75}]{
        \node[anchor=south] at (0.5,0) {\(\mathcal{F}\)};
        \node[anchor=south] at (5.5,0) {\(\mathcal{F}'\)};

        \filldraw[green] (0,0) rectangle +(1,-1);
        \node at (0.5,-0.5) {\(A\)};
        \filldraw[red] (0,-2) rectangle +(1,-1);
        \node at (0.5,-2.5) {\(A\)};
        \filldraw[gray] (0,-4) rectangle +(1,-1);
        \node at (0.5,-4.5) {\(A\)};
        \filldraw[green] (5,0) rectangle +(1,-1);
        \node at (5.5,-0.5) {\(A\)};
        \filldraw[red] (5,-2) rectangle +(1,-1);
        \node at (5.5,-2.5) {\(A\)};
        \filldraw[gray] (5,-4) rectangle +(1,-1);
        \node at (5.5,-4.5) {\(A\)};

        \foreach \i in {-0.5,-2.5,-4.5} \draw[|->,thick] (1.1,\i) -- +(3.8,0);
        \node[anchor=north] at (3,-0.5) {\(T + \tau \frac{T_{igni} - T_{i}}{[\mathcal{T}]_{i,j}}\)};
        \node[anchor=north] at (3,-2.5) {\(T + 25 \tau\)};
        \node[anchor=north,scale=0.9] at (3,-4.5) {\(\displaystyle\sum_{N \in V(A)^{*}} \frac{T_{N}}{|V(A)^{*}|}\)};

        \draw[step=1] (0,0) grid +(1,-1);
        \draw[step=1] (0,-2) grid +(1,-1);
        \draw[step=1] (0,-4) grid +(1,-1);
        \draw[step=1] (5,0) grid +(1,-1);
        \draw[step=1] (5,-2) grid +(1,-1);
        \draw[step=1] (5,-4) grid +(1,-1);
      }

      \column{.4\textwidth}
      \textbf{Boucle d'actualisation de \(q\)}
      \bigbreak
      \tikz[scale=0.75,every node/.style={scale=0.75}]{
        \node[anchor=south] at (0.5,0) {\(\mathcal{F}\)};
        \node[anchor=south] at (4.5,0) {\(\mathcal{F}'\)};

        \filldraw[green] (0,0) rectangle +(1,-2);
        \foreach \i in {-0.5,-1.5} \node at (0.5,\i) {\(q = 0\)};
        \filldraw[red] (0,-2) rectangle +(1,-2);
        \foreach \i in {-2.5,-3.5} \node at (0.5,\i) {\(q = 1\)};
        \filldraw[gray] (0,-4) rectangle +(1,-1);
        \node at (0.5,-4.5) {\(q = 2\)};
        \filldraw[green] (4,0) rectangle +(1,-1);
        \node at (4.5,-0.5) {\(q = 0\)};
        \filldraw[red] (4,-1) rectangle +(1,-2);
        \foreach \i in {-1.5,-2.5} \node at (4.5,\i) {\(q = 1\)};
        \filldraw[gray] (4,-3) rectangle +(1,-2);
        \foreach \i in {-3.5,-4.5} \node at (4.5,\i) {\(q = 2\)};

        \foreach \i in {-0.5,...,-4.5} \draw[|->,thick] (1.1,\i) -- +(2.8,0);
        \node[anchor=north] at (2.5,-0.5) {si \(T < T_{igni}\)};
        \node[anchor=south] at (2.5,-1.5) {si \(T \geq T_{igni}\)};
        \node[anchor=north,scale=0.9] at (2.5,-2.5) {si \(T < 5000 + T_{igni}\)};
        \node[anchor=south,scale=0.9] at (2.5,-3.5) {si \(T \geq 5000 + T_{igni}\)};

        \draw[step=1] (0,0) grid +(1,-5);
        \draw[step=1] (4,0) grid +(1,-5);
      }
    \end{columns}
  }
  % 4 - RÉSULTATS
  \section{Résultats}
  % 4.1 - Paramètres fixés, 1 point de départ, sans vent
  \subsection{Paramètres fixés, 1 point de départ, sans vent}
  \frame{
    \frametitle{Résultats}
    \framesubtitle{Paramètres fixés, 1 point de départ, sans vent}
    \begin{onlyenv}<1>
      \textbf{Détails}
      \bigbreak
      \includegraphics[width=\textwidth]{imgs/x.png}
    \end{onlyenv}\begin{onlyenv}<2>
      \textbf{\(t = 0\)}
      \bigbreak
      \includegraphics[width=\textwidth]{imgs/x_0.png}
    \end{onlyenv}\begin{onlyenv}<3>
      \textbf{\(t = 60\), Modèle 1}
      \bigbreak
      \includegraphics[width=\textwidth]{imgs/x_60,1.png}
    \end{onlyenv}\begin{onlyenv}<4>
      \textbf{\(t = 60\), Modèle 2}
      \bigbreak
      \includegraphics[width=\textwidth]{imgs/x_60,2.png}
    \end{onlyenv}\begin{onlyenv}<5>
      \textbf{\(t = 240\), Modèle 1}
      \bigbreak
      \includegraphics[width=\textwidth]{imgs/x_240,1.png}
    \end{onlyenv}\begin{onlyenv}<6>
      \textbf{\(t = 240\), Modèle 2}
      \bigbreak
      \includegraphics[width=\textwidth]{imgs/x_240,2.png}
    \end{onlyenv}
  }
  % 4.2 - Paramètres fixés, 5 points de départ, avec vent
  \subsection{Paramètres fixés, 5 points de départ, avec vent}
  \frame{
    \frametitle{Résultats}
    \framesubtitle{Paramètres fixés, 5 points de départ, avec vent}
    \begin{onlyenv}<1>
      \textbf{Détails}
      \bigbreak
      \includegraphics[width=\textwidth]{imgs/p.png}
    \end{onlyenv}\begin{onlyenv}<2>
      \textbf{\(t = 0\)}
      \bigbreak
      \includegraphics[width=\textwidth]{imgs/p_0.png}
    \end{onlyenv}\begin{onlyenv}<3>
      \textbf{\(t = 60\), Modèle 1}
      \bigbreak
      \includegraphics[width=\textwidth]{imgs/p_60,1.png}
    \end{onlyenv}\begin{onlyenv}<4>
      \textbf{\(t = 60\), Modèle 2}
      \bigbreak
      \includegraphics[width=\textwidth]{imgs/p_60,2.png}
    \end{onlyenv}\begin{onlyenv}<5>
      \textbf{\(t = 240\), Modèle 1}
      \bigbreak
      \includegraphics[width=\textwidth]{imgs/p_240,1.png}
    \end{onlyenv}\begin{onlyenv}<6>
      \textbf{\(t = 240\), Modèle 2}
      \bigbreak
      \includegraphics[width=\textwidth]{imgs/p_240,2.png}
    \end{onlyenv}
  }
  % 4.3 - Paramètres aléatoires, 5 points de départ, avec vent
  \subsection{Paramètres aléatoires, 5 points de départ, avec vent}
  \frame{
    \frametitle{Résultats}
    \framesubtitle{Paramètres aléatoires, 5 points de départ, avec vent}
    \begin{onlyenv}<1>
      \textbf{Détails}
      \bigbreak
      \includegraphics[width=\textwidth]{imgs/a.png}
    \end{onlyenv}\begin{onlyenv}<2>
      \textbf{\(t = 0\)}
      \bigbreak
      \includegraphics[width=\textwidth]{imgs/a_0.png}
    \end{onlyenv}\begin{onlyenv}<3>
      \textbf{\(t = 60\), Modèle 1}
      \bigbreak
      \includegraphics[width=\textwidth]{imgs/a_60,1.png}
    \end{onlyenv}\begin{onlyenv}<4>
      \textbf{\(t = 60\), Modèle 2}
      \bigbreak
      \includegraphics[width=\textwidth]{imgs/a_60,2.png}
    \end{onlyenv}\begin{onlyenv}<5>
      \textbf{\(t = 240\), Modèle 1}
      \bigbreak
      \includegraphics[width=\textwidth]{imgs/a_240,1.png}
    \end{onlyenv}\begin{onlyenv}<6>
      \textbf{\(t = 240\), Modèle 2}
      \bigbreak
      \includegraphics[width=\textwidth]{imgs/a_240,2.png}
    \end{onlyenv}
  }

  \section*{Table des matières}
  \frame[plain]{
    \frametitle{Table des matières}
    \tableofcontents
  }
  \appendix
  \AtBeginSection{}
  \frame[plain]{
    \frametitle{\appendixname}
    \tableofcontents[firstsection=5]
  }
  % 0 - MODULES
  \section*{Modules}
  \frame[containsverbatim]{
    \frametitle{Modules}
    \begin{lstlisting}
import numpy as np
from scipy.integrate import quad
from matplotlib import pyplot as plt
from matplotlib.colors import ListedColormap, BoundaryNorm
from matplotlib.animation import FuncAnimation
from math import ceil
from random import random, randint
from time import time
from types import LambdaType
    \end{lstlisting}
  }
  % I - CLASS TREE
  \section[Tree]{class Tree}
  % I.i - constructor
  \subsection{constructor}
  \frame[containsverbatim]{
    \frametitle{Tree constructor}
    \begin{lstlisting}
class Tree :
  def __init__(self, state:int=0, temperature:float=298.15,  grass:float=None, humidity:float=None, forest:'Forest'=None) :
    self.state = state
    self.temperature = temperature
    self.grass = grass if not grass is None and grass >= 0 else random() * 0.5
    self.humidity = humidity if not humidity is None and 0 <= humidity <= 1 else random()
    if (not forest is None) and isinstance(forest[0], Forest) :
      self.Forest = forest[0]
      self.location = forest[1], forest[2]
    else :
      self.Forest, self.location = None, None
    self._neighbours = None
    \end{lstlisting}
  }
  % I.ii - neighbours
  \subsection{neighbours}
  \frame[containsverbatim]{
    \frametitle{Tree.neighbours}
    \begin{lstlisting}
def neighbours(self) -> list :
  if self.Forest is None :
    raise ValueError('self is not in a Forest')
  if self._neighbours is None :
    i, j = self.location
    self._neighbours = []
    for c in [[i-1,j], [i-1,j+1], [i,j+1], [i+1,j+1], [i+1,j], [i+1,j-1], [i,j-1], [i-1,j-1]] : # N,NE,E,SE,S,SW,W,NW
      if 0 <= c[0] < self.Forest.dim and 0 <= c[1] < self.Forest.dim :
        self._neighbours.append(self.Forest[c[0],c[1]])
      else :
        self._neighbours.append(None)
  return self._neighbours
    \end{lstlisting}
  }
  % I.iii - windDirCalc
  \subsection{windDirCalc}
  \frame[containsverbatim, label=windDirCalc1]{
    \frametitle{Tree.windDirCalc}
    \framesubtitle{Corps principal}
    \begin{lstlisting}
def windDirCalc(self, origin:tuple) -> float :
  if self.Forest is None :
    raise ValueError('self is not in a Forest')
  if self.Forest.wind == -1 : # Wind : null
    return 0
  tree = (self.location[0] - origin[0], self.location[1] - origin[1]) # Coordinates of self relative to origin
  if tree == (0,0) : # self is origin
    return 0
  %*\hyperlink{windDirCalc2}{\beamergotobutton{[\dots]}}*
  return (tree[0] * wind[0] + tree[1] * wind[1]) / (((tree[0] ** 2 + tree[1] ** 2) * (wind[0] ** 2 + wind[1] ** 2))  ** .5) # Dot product / Norms product
    \end{lstlisting}
  }
  \frame[containsverbatim, label=windDirCalc2]{
    \frametitle{Tree.windDirCalc}
    \framesubtitle{Disjonction de cas}
    \begin{lstlisting}
def windDirCalc(self, origin:tuple) -> float :
  %*\hyperlink{windDirCalc1}{\beamerreturnbutton{[\dots]}}*
  wind = (0,0)
  if self.Forest.wind == 0 : # Wind : N
    wind = (-1,0)
  elif self.Forest.wind == 1 : # Wind : NE
    wind = (-1,1)
  elif self.Forest.wind == 2 : # Wind : E
    wind = (0,1)
  elif self.Forest.wind == 3 : # Wind : SE
    wind = (1,1)
  elif self.Forest.wind == 4 : # Wind : S
    wind = (1,0)
  elif self.Forest.wind == 5 : # Wind : SW
    wind = (1,-1)
  elif self.Forest.wind == 6 : # Wind : W
    wind = (0,-1)
  else : # Wind : NW
    wind = (-1,-1)
  %*\hyperlink{windDirCalc1}{\beamerreturnbutton{[\dots]}}*
    \end{lstlisting}
  }
  % II - CLASS FOREST
  \section[Forest]{class Forest}
  % II.i - constructor
  \subsection{constructor}
  \frame[containsverbatim]{
    \frametitle{Forest constructor}
    \begin{lstlisting}
class Forest :
  def __init__(self, dim:int, start:list or LambdaType=[], wind:int=-1) :
    self.__Forest = np.array([[Tree(forest=[self,i,j]) for j in range(dim)] for i in range(dim)])
    self.dim = dim
    self.wind = wind
    self.start = []
    if isinstance(start, LambdaType) :
      self.start = start(dim)
    elif isinstance(start, list) :
      self.start = start
    for (i,j) in [(i,j) for (i,j) in self.start if 0 <= i < dim and 0 <= j < dim] :
      self.__Forest[i,j].state = 1
      self.__Forest[i,j].temperature = 573.15
      self.__Forest[i,j].humidity = 0
    self._Wind = None
    self._State = None
    self._Temperature = None
    self._Grass = None
    self._Humidity = None
    \end{lstlisting}
  }
  % II.ii - rndStart, distCalc
  \subsection{rndStart, distCalc}
  \frame[containsverbatim]{
    \frametitle{Forest.rndStart, Forest.distCalc}
    \begin{lstlisting}
def rndStart(n:int=1) -> '(int) -> list' :
  def f(n, dim) :
    start = []
    while len(start) < n :
      i, j = randint(0, dim), randint(0, dim)
      if not (i,j) in start :
        start.append((i,j))
    return start
  return lambda dim : f(n, dim)
    \end{lstlisting}\begin{lstlisting}
def distCalc(a:tuple, b:tuple) -> float :
  return 3 * (((b[0] - a[0]) ** 2 + (b[1] - a[1]) ** 2) ** .5)
    \end{lstlisting}
  }
  % III - CLASS WILDFIRE
  \section[Wildfire]{class Wildfire}
  % III.i - constructor
  \subsection{constructor}
  \frame[containsverbatim]{
    \frametitle{Wildfire constructor}
    \begin{lstlisting}
class Wildfire :
  def __init__(self, forest:'Forest', model:'(Wildfire) -> Forest', dt:float=1) :
    if not isinstance(forest, Forest) :
      raise TypeEror('forest is not a Forest instance')
    if len(forest.start) == 0 :
      raise ValueError('forest has no start tree')
    self.__History = [forest]
    self.model = model
    self.dt = dt
    self._physics = None
    \end{lstlisting}
  }
  % III.ii - physics
  \subsection{physics}
  \frame[containsverbatim, label=physics1]{
    \frametitle{Wildfire.physics}
    \framesubtitle{Corps principal}
    \begin{lstlisting}
def physics(self) -> dict :
  if self._physics is None :
    # Compute the speeds of the wind without and with wind
    v0 = lambda d,h: (5.670e-8 * 1300.15 ** 4 * d) / (2 * 0.2 * (1200 * (573.15 - 298.15) + 2256e3 * h))
    v1 = lambda d,h: v0(d,h) * 0.90 * 2 * (1 + np.sin(0.35) - np.cos(0.35)) / d
    # Compute the time before the trees will burn and where the flames will come from
    Time = np.array([[0 for j in range(self[0].dim)] for i in range(self[0].dim)])
    flameOrigin = np.array([[-1 for j in range(self[0].dim)] for i in range(self[0].dim)])
    %*\hyperlink{physics2}{\beamergotobutton{[\dots]}}*
    self._physics = {
      'v0': v0,
      'v1': v1,
      'Time': Time,
      'flameOrigin': flameOrigin
    }
  return self._physics
    \end{lstlisting}
  }
  \frame[containsverbatim, label=physics2]{
    \frametitle{Wildfire.physics}
    \framesubtitle{Boucle de calcul de Time et flameOrigin (1/2)}
    \begin{lstlisting}
def physics(self) -> dict :
  %*\hyperlink{physics1}{\beamerreturnbutton{[\dots]}}*
  for i in range(self[0].dim) :
    for j in range(self[0].dim) :
      if (i,j) in self[0].start :
        Time[i,j] = 0
        flameOrigin[i,j] = self[0].start.index((i,j))
      else :
        min, smin = float('inf'), -1
        %*\hyperlink{physics3}{\beamergotobutton{[\dots]}}*
        Time[i,j] = ceil(min)
        flameOrigin[i,j] = smin
  %*\hyperlink{physics1}{\beamerreturnbutton{[\dots]}}*
    \end{lstlisting}
  }
  \frame[containsverbatim, label=physics3]{
    \frametitle{Wildfire.physics}
    \framesubtitle{Boucle de calcul de Time et flameOrigin (2/2)}
    \begin{lstlisting}
def physics(self) -> dict :
        %*\hyperlink{physics2}{\beamerreturnbutton{[\dots]}}*
        for (si,sj) in self[0].start :
          f = None
          if self[0].wind == -1 :
            f = lambda t: Forest.distCalc((i,j), (si,sj)) / v0(self[0][int(t*i+(1-t)*si),int(t*j+(1-t)*sj)].grass, self[0][int(t*i+(1-t)*si),int(t*j+(1-t)*sj)].humidity) # Distance between the tree and the start point #s / Speed without wind
          else :
            f = lambda t: Forest.distCalc((i,j), (si,sj)) / ((1.3 + self[0][i,j].windDirCalc((si,sj))) * v1(self[0][int(t*i+(1-t)*si),int(t*j+(1-t)*sj)].grass, self[0][int(t*i+(1-t)*si),int(t*j+(1-t)*sj)].humidity)) # Distance between the tree and the start point #s / (1.3 + Cosine of angle) * Speed with wind
          t = quad(f,0,1)[0]
          if t < min :
            min = t
            smin = self[0].start.index((si,sj))
        %*\hyperlink{physics2}{\beamerreturnbutton{[\dots]}}*
    \end{lstlisting}
  }
  % IV - MODÈLES
  \section{Modèles}
  % IV.i - Premier modèle
  \subsection{Premier modèle}
  \frame[containsverbatim]{
    \frametitle{Premier modèle}
    \begin{lstlisting}
def Model1(wildfire:'Wildfire') -> 'Forest' :
  nextForest = Forest.copy(wildfire[-1])
  for T in nextForest :
    if T.state == 1 : # State : Burning
      T.temperature += wildfire.dt * 25
      for N in [N for N in T.neighbours if not N is None] :
        N.temperature += wildfire.dt * 15 * (1 + N.windDirCalc(T.location))
    elif T.state == 2 : # State : Burnt
      avg, n = 0, 0
      for N in [N for N in T.neighbours if not N is None] :
        avg += wildfire[-1][N.location].temperature
        n += 1
      T.temperature = avg / n
  for T in nextForest :
    if T.state == 0 and T.temperature >= 573.15 :
      T.state = 1
    elif T.state == 1 and T.grass >= 5573.15 :
      T.state = 2
  return nextForest
    \end{lstlisting}
  }
  % IV.ii - Second modèle
  \subsection{Second modèle}
  \frame[containsverbatim]{
    \frametitle{Second modèle}
    \begin{lstlisting}
def Model2(wildfire:'Wildfire') -> 'Forest' :
  nextForest = Forest.copy(wildfire[-1])
  for T in nextForest :
    if T.state == 0 : # State : Normal
      T.temperature += wildfire.dt * (573.15 - wildfire[0][T.location].temperature) / wildfire.physics['Time'][T.location]
    elif T.state == 1 : # State : Burning
      T.temperature += wildfire.dt * 25
    else : # State : Burnt
      avg, n = 0, 0
      for N in [N for N in T.neighbours if not N is None] :
          avg += wildfire[-1][N.location].temperature
          n += 1
      T.temperature = avg / n
  for T in nextForest :
    if T.state == 0 and T.temperature >= 573.15 :
        T.state = 1
    elif T.state == 1 and T.grass >= 5573.15 :
      T.state = 2
  return nextForest
    \end{lstlisting}
  }
\end{document}
